\chapter{Introduction}

This document has as purpose the explanation of the concepts in the problem presented in the MOPTA challenge, the methodology, techniques and assumptions used to implement a model and the mathematical implementation of said model.

In order to understand this document, it is necessary to first read and understand the problem at hand. The problem consists on a logistics and production problem where the product quality decays over time. This has implications such as that there are lower and upper limits for the time that needs to pass between production and consumption. Routing also needs to be used to transport efficiently the products between nodes.

More detail about the problem can be found in the \href{http://coral.ie.lehigh.edu/~mopta/AIMMS_MOPTA_case_2017.pdf}{following document}.

This document contains three chapters. In the first one, some definitions are made about production, transportation and demand as well as some grouping assumptions of these.

Te second chapter focuses on the Mathematical Formulation for the model that solves and returns a solution for this problem.

Finally, a summary of the tests that were done and a brief representation of a solution.
